\documentclass[11pt]{article}

\usepackage{graphicx}% Include figure files
\usepackage{dcolumn}% Align table columns on decimal point
\usepackage{bm}% bold math
%\usepackage{hyperref}% add hypertext capabilities
%\usepackage[mathlines]{lineno}% Enable numbering of text and display math
%\linenumbers\relax % Commence numbering lines

%\usepackage[showframe,%Uncomment any one of the following lines to test
%%scale=0.7, marginratio={1:1, 2:3}, ignoreall,% default settings
%%text={7in,10in},centering,
%%total={6.5in,8.75in}, top=1.2in, left=0.9in, includefoot,
%%height=10in,a5paper,hmargin={3cm,0.8in},
%]{geometry}

\usepackage{longtable}
\usepackage{lipsum}
\usepackage{color}
\usepackage{amsmath}
\usepackage{amssymb}
\usepackage{caption}
\usepackage{subcaption}
\usepackage{hyperref}
\usepackage{ulem}
\hypersetup{
    colorlinks=true,
    linkcolor=blue,
    filecolor=magenta,      
    urlcolor=cyan,
}

\newcommand{\comment}[1]{}

\setlength{\marginparwidth}{0.75in}
\newcommand{\MarginPar}[1]{\marginpar{%
\vskip-\baselineskip %raise the marginpar a bit
\raggedright\tiny\sffamily
\hrule\smallskip{\color{red}#1}\par\smallskip\hrule}}

\newcommand{\jbb}[1]{{\bf[{\color{red}{JBB:  #1}}]}}
\newcommand{\msd}[1]{{\bf[{\color{green}{MSD:  #1}}]}}
\definecolor{drkgrn}{rgb}{0.043,0.341,0.2274}
\newcommand{\rwg}[1]{{\bf[{\color{drkgrn}{RWG:  #1}}]}}


\newcommand{\HeatFlux}{\boldsymbol{\mathcal{Q}}}
\newcommand{\SpeciesFlux}{\boldsymbol{\mathcal{F}}}
\newcommand{\SpeciesFluxB}{\bar{\boldsymbol{\mathcal{F}}}}
\newcommand{\StressTensor}{\boldsymbol{\Pi}}
\newcommand{\ShearViscosity}{\eta}
\newcommand{\BulkViscosity}{\kappa}
\newcommand{\ThermalConductivity}{\lambda}
\newcommand{\EntropyProduction}{\mathfrak{v}}
\newcommand{\OnsagerMatrix}{\boldsymbol{\mathfrak{L}}}
\newcommand{\OnsagerMatrixB}{\bar{\boldsymbol{\mathfrak{L}}}}
\newcommand{\OnesVector}{\mathbf{u}}
\newcommand{\ChemicalSpeciesLabel}{\mathfrak{M}}
\newcommand{\StochioPM}{\nu}

\newcommand{\mbar}{\overline{m}}
\newcommand{\wbar}{\overline{W}}
\newcommand{\Jbar}{\bar{\mathbf{J}}}
\newcommand{\Xbar}{\bar{\mathbf{X}}}
\newcommand{\Lbar}{\bar{\mathbf{L}}}
\newcommand{\Bbar}{\bar{\mathbf{B}}}
\newcommand{\Wbar}{\bar{\mathcal{Z}}}
\newcommand{\Wcbar}{\bar{\mathcal{W}}}
\newcommand{\Bcbar}{\bar{\mathcal{B}}}
\newcommand{\lbar}{\bar{\mathbf{l}}}
\newcommand{\xibar}{\bar{\xi}}
\newcommand{\DonevDiffusion}{{D}}
\newcommand{\half}{\frac{1}{2}}

\begin{document}
\begin{center}
{\bf
SRK Implementation
}

\vspace{\baselineskip}
John B. Bell \\
Alejandro Garcia \\
Ray Grout \\
Emmanuel Motheau \\
Andrew Nonaka \\
Shashank Yellapantula

\vspace{\baselineskip}
April 19, 2017
\end{center}

\section{Things to compute}
Need software to be able to compute the following things; the below is a template for what needs to be implemented. 

Derivatves of attractive parameters:
\[
\frac{\partial a_m}{\partial Y_k} \quad \frac{\partial a_m}{\partial T} \quad \frac{\partial b_m}{\partial Y_k} \quad \frac{\partial^2 a_m}{\partial T^2}
\]

Derivatives of state variables
\[
\quad \frac{\partial p}{\partial \tau}, \quad \frac{\partial p}{\partial T}
\]

Mixing rules:
\[
b_m(Y_i,b_i) \qquad a_m(Y_i,a_i)
\]

Evaluation of energies for individual species:
\[
h_k(Y, T, \rho) = h_k^{id}(T) + h_k^{dep}(Y,T,\rho)
\]
\[
e_k(Y, T, \rho) = e_k^{id}(T) + e_k^{dep}(Y,T,\rho)
\]

Mixture energies:
\[
e_m(Y, T, \rho) = \left[e_m^{id} = \sum_k Y_Ke_k^{id}\right] + e_m^{dep}(Y,T,\rho)
\]
\[
h_m(Y, T, \rho) = \left[h_m^{id} = \sum_k Y_Kh_k^{id}\right] + h_m^{dep}(Y,T,\rho)
\]
Specific heats
\[
c_p 
\]
\[
c_v
\]
Routines to compute T given the state and either internal energy or enthalpy via a solve using either Newton or Secant method (start with Secant):
\[
T(e_m,Y,\rho), \quad T(h_m,Y,\rho)
\]

Sound speed:
\[
a
\]

Pressure:
\[
p = p^{id}(\rho, Y, T) + \Phi,  \qquad \Phi(\rho, Y, T)
\]

Chemical potentials:
\[
\mu_k = \left(\frac{\partial f}{\partial Y_k}\right)_{\tau,T,Y}
\]

Chemical acitivities $A_k$ for use in reactions

Transport changes --- compute new driving force: 
\MarginPar{Need to check this}
\[
\left(\frac{\partial \tau}{\partial Y}\right)_{T,p}
\]
Use derivatives to compute $\theta$ and hence $\phi$.

\section{Giovangigli approach to deriving formulae}

What we would like to do is start with an equation of state and derive other quantities we need that
are thermodynamically consistent.  The EOS doesn't nail things down completely.  A way to get around that
(which is done by Giovangigli in his nonideal papers) is to
say
\[
p = p^{id} + \Phi
\]
where $p^{id}$ is the ideal gas part and assume that as density goes to zero, the EOS becomes
ideal.  In other words we look at deviation form ideal.
What Giovangigli does is to write thermodynamic quantities as
\[
\Psi = \Psi^{id} - \int_\tau^\infty \frac{\partial \Psi}{\partial \tau} d \tau
\]

For internal energy from the Helmholtz equation
\begin{equation}
\frac{\partial e}{\partial \tau} = T^2 \frac{\partial (\Phi/T)} {\partial T}
\end{equation}
If we use the above and note that $p^{id}/T$ is independent of $T$ then
\begin{equation}
e = e^{id}- \int_\tau^\infty  T^2 \frac{\partial (\Phi/T)} {\partial T} d\tau
\label{eq:energy_gen}
\end{equation}

For entropy of mixture, $s_m$,
\begin{equation}
s_m = s_m^{id} -
-\int_\tau^{\infty} \frac{\partial (\Phi)} {\partial T} d\tau ' 
\end{equation}


From free energy we also have
\[
\mu_k = \frac{\partial f}{\partial Y_k} = 
 \frac{\partial e_m}{\partial Y_k} - 
 T\frac{\partial s_m}{\partial Y_k} 
\]
so that
\begin{equation}
\mu_k =
 \mu_k^{id} + \int_\tau^\infty \frac{\partial (\Phi)} {\partial Y_k} d\tau
\end{equation}

Enthalpy of the mixture is given by $h_m = e_m + P \tau$.  Partial enthalpies are
\[
h_k = \left. \frac{\partial h}{\partial Y_k} \right|_{T,p,Y_{k\ne l}}
\]
The natural form given by the Giovangigli approach gives things as a function of $\rho$, $T$ and $Y_k$.
This can be address using
\begin{equation}
\frac{\partial h}{\partial (T,Y,p)}
= \frac{\partial h}{\partial (T,Y,\rho)}
\frac{\partial (T,Y,\rho)}{\partial (T,Y,p)}
= \frac{\partial h}{\partial (T,Y,\rho)}
\left (\frac{\partial (T,Y,p)}{\partial (T,Y,\rho)} \right)^{-1}
\label{eq:calc_cov}
\end{equation}

\section{Specialization for SRK}
\label{sec:eos}


\[
p = p^{id} + \Phi
\]

\[
p^{id} = R T \sum \frac{Y_k}{W_k} \frac{1}{\tau}
\]
\[
\label{eq:pSRK}
p = R T \sum \frac{Y_k}{W_k} \frac{1}{\tau - b_m} - \frac{a_m}{\tau(\tau + b_m)}
\]
\[
\Phi
= R T \sum \frac{Y_k}{W_k} \frac{b_m}{\tau(\tau -b_m)} - \frac{a_m}{\tau (\tau + b_m}
\]
Mixing rules $a_m = a_m(T, Y_k)$ and $b_m = b_m(Y_k)$
with mixing rules
\[
a_m = \sum_{ij} Y_i Y_j \alpha_i \alpha_j \;\;\;  b_m = \sum_k Y_k b_k
\]
Note that the $a_j$ and $b_i$ are scaled by molecular weights relative to what is often used
in the thermodynamics literature.
\MarginPar{need to fix these}
Typically, $a_i$ and $b_i$ are defined using the critical quantities and temperature dependent correction factors~\cite{poling2001properties}:
\begin{align}
\alpha_i & = \sqrt{a_i} \nonumber \\
a_i(T) &= a_i^{*}\left(T_{c,i},P_{c,i}\right) \bar{a}_i (T/T_{c,i})  \nonumber \\
       &= 0.42748 \frac{\left(R T_{c,i} \right)^2}{W_i^2 p_{c,i}} \bar{a}_i \left(T/T_{c,i}\right)  \nonumber \\
 b_i &= 0.08664 \frac{R T_{c,i}}{W_i p_{c,i}}   
       \label{eq:abformEOS}
\end{align}
where
\[
\bar{a}_i (T/T_{c,i}) = \left(1 + \mathcal{A} \left[ f\left( \omega_i \right) \left(1-\sqrt{T/T_{c,i}} \right ) \right] \right)^2
\]
%\jbb{One would like to use a mollified version of the $\alpha$ to smooth behavior near critical point}
where $\omega_i$ are the acentric terms.
Here $\mathcal{A}$ is a smoothing function to make derivatives smmoth at $T_{c,i}$
\[
f\left( \omega_i \right) = 0.48508 + 1.5517 \omega_i - 0.151613 \omega_{i}^2
\]

\[
e_m = \sum_k Y_k e_k^{id} + \left( T  \frac{\partial a_m}{\partial T}  - a_m \right)
\frac{1}{b_m} ln ( 1 + \frac{b_m}{\tau})
\]
To define the entropy 
\begin{equation}
s_m = \sum_k Y_k s_k^{id} - 
 \sum_k \frac{Y_k R}{W_k} ln \left(\frac{\tau}{\tau-b_m} \right)
+ \frac{\partial a_m}{\partial T} \frac{1}{b_m} ln ( 1 + \frac{b_m}{\tau})
\end{equation}
In this form, 
\[
s_k^{id} = s_k^{id,*} - \frac{R}{W_k} ln(\frac{Y_k R T}{W_k \tau  p^{st}})
\]
where
\[
s_k^{id,*} = s_k^{st} + \int_{T^{st}}^T  c_p(T')/T' dT'
\]
where $p^{st}$ is the standard reference pressure
So when you put this together you get
\[
s_m = \sum_k Y_k s_k^{id,*} - \sum_k \frac{Y_k R}{W_k} ln  \left( \frac{Y_k R T}{W_k (\tau -b_m) p^{st}}   \right )
+ \frac{\partial a_m}{\partial T} \frac{1}{b_m} ln ( 1 + \frac{b_m}{\tau})
\]
From definition of enthalpy,
\[
\label{eq:hmSRK}
h_m = \sum_k Y_k h_k^{id} + \left ( T \frac{\partial a_m}{\partial T} - a_m \right)
\frac{1}{b_m} ln ( 1 + \frac{b_m}{\tau})
+ 
R T \sum \frac{Y_k}{W_k} \frac{b_m}{\tau -b_m} - \frac{a_m}{\tau + b_m}
\]

Start with specific heats
\[
c_v = \left( \frac{\partial e_m}{\partial T}\right)_{\tau,Y}
\]
which becomes
\[
c_v = c_v^{id} + T \frac{\partial^2 a_m}{\partial T^2} \frac{1}{b_m} ln ( 1 + \frac{b_m}{\tau})
\]
\[
c_p =  \frac{\partial h_m}{\partial T}
- \frac {\frac{\partial h}{\partial \tau}} 
 {\frac{\partial p}{\partial \tau}}
\frac{\partial p}{\partial T}
\]
where derivatives are with respect $\rho, T, Y_k$ independent variables
\[
\frac{\partial p}{\partial T} = \sum Y_k / W_k  \frac{R}{\tau-b_m} - \frac{\partial a_m}{\partial T} \frac{1}{\tau(\tau +b_m)}
\]
\[
\frac{\partial p}{\partial \tau} =
\sum Y_k / W_k  \frac{R T}{(\tau-b_m)^2} + \frac{a_m (2 \tau + b_m)}{[\tau(\tau +b_m)]^2}
\]

\vspace{\baselineskip}
\[
\frac{\partial h_m}{\partial \tau} =
-(T \frac{\partial a_m}{\partial T}  - a_m )
\frac{1}{\tau(\tau+b_m)}
+ \frac{a_m}{(\tau+b_m)^2}
-\sum Y_k / W_k  \frac{R T b_m}{(\tau-b_m)^2} 
\]
\vspace{\baselineskip}
\[
\frac{\partial h_m}{\partial T} = c_p^{id}
+T \frac{\partial^2 a_m}{\partial T^2} 
\frac{1}{b_m} ln ( 1 + \frac{b_m}{\tau})
- \frac{\partial a_m}{\partial T} 
\frac{1}{\tau+b_m}
+\sum Y_k / W_k  \frac{R b_m}{\tau-b_m} 
\]
Sound speed
\[
a^2 = -\frac{c_p}{c_v} \tau^2  \frac{\partial p}{\partial \tau}
\]

\[
h_k = \frac{\partial h_m}{\partial Y_k }
- \frac {\frac{\partial h}{\partial \tau}} 
 {\frac{\partial p}{\partial \tau}}
 \frac{\partial p}{\partial Y_k}
\]
\begin{align}
\frac{\partial h_m}{\partial Y_k } &=  h_k^{id}
+ (T \frac{\partial^2 a_m}{\partial T \partial Y_k}  - \frac{\partial a_m }{Y_k})
\frac{1}{b_m} ln(1+ \frac{b_m}{\tau})
-(T \frac{\partial a_m}{\partial T}  - a_m ) \left[ \frac{1}{b_m^2} ln(1+ \frac{b_m}{\tau})
+ \frac{1}{b_m(\tau+b_m)}
\right ] \frac{\partial b_m}{\partial Y_k} \nonumber \\
&+ \frac{a_m}{(\tau+b_m)^2}  \frac{\partial b_m}{\partial Y_k}
- \frac{1}{\tau+b_m}  \frac{\partial a_m}{\partial Y_k}
+ 1 / W_k  \frac{R T b_m}{\tau-b_m}+
\sum_i \frac{Y_i}{W_i} R T ( \frac{1}{\tau -b_m} + \frac{b_m}{(\tau-b_m)^2} ) \frac{ \partial b_m}{\partial Y_k} 
\end{align}
\[
 \frac{\partial p}{\partial Y_k}
= 
 R T \frac{1}{W_k} \frac{1}{\tau - b_m} - \frac{\partial a_m}{\partial Y_k} \frac{1}{\tau(\tau + b_m)}
+\left(R T \sum \frac{Y_i}{W_i} \frac{1}{(\tau - b_m)^2} + \frac{a_m}{\tau(\tau + b_m)^2} \right ) \frac{\partial b_m}{\partial Y_k}
\]

For kinetics and transport need chemical potential. For that free energy is given by
\begin{align}
f &= \sum_i Y_i (e_i^{id} - T s_i^{id,*}) +  \sum_i \frac{Y_i R T}{W_i} ln (\frac{Y_i R T}{W_i \tau p^{st}})  \nonumber \\
&+ \sum_i \frac{Y_i R T}{W_i} ln (\frac{\tau}{\tau-b_m}) - \frac{\partial a_m}{\partial T} \frac{1}{b_m}
ln (1+ \frac{b_m}{\tau})  \nonumber \\
 &= \sum_i Y_i (e_i^{id} - T s_i^{id,*}) +  \sum_i \frac{Y_i R T}{W_i} ln (\frac{Y_i R T}{W_i (\tau-b_m) p^{st}} )
- \frac{\partial a_m}{\partial T} \frac{1}{b_m}
ln (1+ \frac{b_m}{\tau})  \nonumber 
\end{align}
Then
\begin{align}
\mu_k &= \frac{\partial f}{\partial Y_k} = 
e_k^{id} - T s_k^{id,*}  + \frac{RT}{W_k} ln (\frac{Y_k R T}{W_k (\tau-b_m) p^{st}})
+ \frac{RT}{W_k} +  \frac{RT}{\wbar} \frac{1}{\tau-b_m} \frac {\partial b_m}{\partial Y_k} \nonumber \\
&- \frac{1}{b_m} ln(1 + \frac{b_m}{\tau}) \frac{\partial a_m}{\partial Y_k}
+ \frac{a_m}{b_m^2} ln(1 + \frac{b_m}{\tau}) \frac{\partial b_m}{\partial Y_k}
- \frac{a_m}{b_m} \frac{1}{\tau+b_m} \frac{\partial b_m}{\partial Y_k}
\end{align}
Letting
\begin{align}
\mu_k^{id,u} &=  e_k^{id} - T s_k^{id,*} + \frac{RT}{W_k}+ \frac{R T}{W_k} ln( \frac{RT}{p^{st}})
 &=  h_k^{id} - T s_k^{id,*} + \frac{R T}{W_k} ln( \frac{RT}{p^{st}})
\end{align}
\begin{align}
\mu_k &= \mu_k^{id,u} + \frac{RT}{\wbar} \frac{1}{\tau-b_m} \frac {\partial b_m}{\partial Y_k}
+ \frac {RT}{W_k} ln ( \frac{Y_k}{W_k (\tau-b_m)}) \nonumber \\
&- \frac{1}{b_m} ln(1 + \frac{b_m}{\tau}) \frac{\partial a_m}{\partial Y_k}
+ \frac{a_m}{b_m^2} ln(1 + \frac{b_m}{\tau}) \frac{\partial b_m}{\partial Y_k}
- \frac{a_m}{b_m} \frac{1}{\tau+b_m} \frac{\partial b_m}{\partial Y_k}
\end{align}
\begin{align}
A_k &=  exp  (  \frac{W_k}{RT}   [
\frac{RT}{\wbar} \frac{1}{\tau-b_m} \frac {b_m}{Y_k}
+ \frac {RT}{W_k} ln ( \frac{Y_k}{W_k (\tau-b_m)}) \nonumber \\
&- \frac{1}{b_m} ln(1 + \frac{b_m}{\tau}) \frac{\partial a_m}{\partial Y_k}
+ \frac{a_m}{b_m^2} ln(1 + \frac{b_m}{\tau}) \frac{\partial b_m}{\partial Y_k}
- \frac{a_m}{b_m} \frac{1}{\tau+b_m} \frac{\partial b_m}{\partial Y_k} ] )
%\right ] \right )
\end{align}


\section{Kinetics}
\label{sec:kinetics}
Chemical potential is given by
\[
\label{eq:NIpotential}
\mu_k = \left( \frac{\partial f}{\partial Y_k}\right)_{\tau,T,Y_j}
\]

For kinetics you work with rescaled chemical potentials
\[
\hat{\mu} = \frac{1}{k_B T} \mathcal{M} \mu \;\;\;  \mathrm{or, equivalently} \;\;\;  
\hat{\mu}_k = \frac{W_k}{R T} \mu_k
\]
Molar production rates are given by
\[
\Omega_i = \sum_j (\nu_{ij}^b - \nu_{ij}^f) \mathcal{R}_j
\]
where 
\[
\mathcal{R}_j = k_j^s \left [ exp(\sum_k \nu_{jk}^f \hat{\mu}_k) - exp (\sum_k  \nu_{jk}^b \hat{\mu}_k) \right ] 
\]
where $k_j^s$ is the symmetric reaction rate.

Here we express the rates in terms of a variant of activities, $a_k$ defined below.
This is more synoptic, see below for style of reasoning

Write recaled ideal potential  $\hat{\mu}_k^{id}$ as
The most natural was is to write it as
\begin{equation}
\hat{\mu}_k^{id} = \frac{W_k}{R T} (h_k^{id} - T s_k^{id,*} )+ ln(RT/p_0) + ln([X_k])
\end{equation}
where $[X_k]$ is the molar concentration.
This leads to an alternative form of the pure component ideal chemical potential
\begin{equation}
\hat{\mu}_k^{id,u} = \frac{W_k}{R T} (h_k^{id} - T s_k^{id,*} )+ ln(RT/p_0) 
\label{eq:chem_uid_conc}
\end{equation}

Now define
\begin{equation}
A_k = exp[\hat{\mu}_k - \hat{\mu}_k^{u,id}]
\label{eq:deviate_ideal}
\end{equation}
The $A_k$ are called the chemical activities.
In the ideal gas limit these become the molar concentrations $[X_k]$.
%This is same as $\gamma_k$ above up to a possible scaling . . . need to check that.
%With this definition, in the ideal gas limit, $\gamma_k = 1$.

With this setup we write the reactions in the more traditional way
\begin{equation}
\mathcal{R}_j = k_j^f \prod_k (A_k)^{\nu_{jk}^f}
- k_j^b \prod_k (A_k)^{\nu_{jk}^b}
\end{equation}
In this form the forward and reverse rates are related
\begin{equation}
 k_j^f = k_j^b K_j^{eq}
\end{equation}
where
\begin{equation}
K_j^{eq} = 
exp( (\nu_{jk}^f - \nu_{jk}^b) \hat{\mu}_k^{id,u})
\label{eq:kin_equil_trad}
\end{equation}
This is what's in chemkin but that break it into two pieces.

We can use $k_j^f$ in traditional Arrehnius form with this definition of equilibrium and all is thermodynamically
consistent.  Can also try working in symmteric form with
\[
k_j^s
=  k_j^f 
%\left ( \frac{p_0}{R T}  \right )^{\sum \nu_{jk}^f }exp( -\nu_{jk}^f \hat{\mu}_k^{u,id})
exp( - \sum_k \nu_{jk}^f \hat{\mu}_k^{u,id})
=  k_j^b 
%\left ( \frac{p_0}{R T}  \right )^{\sum \nu_{jk}^f }exp( -\nu_{jk}^f \hat{\mu}_k^{u,id})
exp( -  \sum_k \nu_{jk}^b \hat{\mu}_k^{u,id})
\]

\section{Species transport -- NEW}

When $\mu$ (vector of chemical potential) is written as a funcion of $T,Y,p$ then
from Onsager relations
the diffusive driving force is given by
\[
d = \frac{\wbar}{RT} \mathcal{Y} \; \nabla_T \; \mu
\]
where the transport is given by
\[
\SpeciesFlux = - \rho \mathcal{Y} \mathcal{D} ( d + \mathcal{X} \frac{ \tilde{\chi}}{T} \nabla T )
\]
However, we are given $\mu$ as a function of $T,Y,\tau$ so we need to use change of variables formulae
to write as
\[
\nabla_T \; \mu_k = 
\frac{\frac{\partial \mu_k}{\partial \tau}}
{\frac{\partial p}{\partial \tau}} \nabla p
+
\left (
\frac{\partial \mu_k}{\partial Y} -
\frac{\frac{\partial \mu_k}{\partial \tau}}
{\frac{\partial p}{\partial \tau}} 
\frac{\partial p}{\partial Y} 
\right ) \nabla Y
\]
Thus we need to compute
\[
\frac{\partial \mu_k}{\partial Y_j}
\]
and
\[
\frac{\partial \mu_k}{\partial \tau}
\]
First we have
\begin{align}
\frac{\partial \mu_k}{\partial Y_j} &=  \frac{RT}{W_k Y_k} \delta_{jk} + \frac{RT}{\tau-b_m} \left(
\frac{1}{W_k} \frac{\partial b_m}{\partial Y_j}+\frac{1}{W_j} \frac{\partial b_m}{\partial Y_k} \right )
+ \frac{RT}{\wbar} \frac{1} {(\tau - b_m)^2} \frac{\partial b_m}{\partial Y_k} \frac{\partial b_m}{\partial Y_j}  \nonumber \\
&- \frac{1}{b_m} ln ( 1 + \frac{b_m}{\tau} ) \frac{\partial^2 a_m }{\partial Y_k \partial Y_j}
+ \left ( \frac{1}{b_m^2} ln ( 1 + \frac{b_m}{\tau} ) - \frac{1}{b_m} \frac{1}{\tau+b_m}  \right)
\frac{\partial a_m }{ \partial Y_j}
\frac{\partial b_m }{\partial Y_k }
\nonumber \\
&+\left( \frac{1}{b_m^2} ln ( 1 + \frac{b_m}{\tau} ) \frac{\partial a_m}{\partial Y_k} - \frac{1}{b_m} \frac{1}{\tau+b_m} \right )
\frac{\partial b_m }{\partial Y_k } \nonumber \\
&+ \left( \frac{-2 a_m}{b_m^3} ln ( 1 + \frac{b_m}{\tau} ) + \frac{a_m}{b_m^2} \frac{1}{\tau+b_m} \right )
\frac{\partial b_m }{\partial Y_j } \frac{\partial b_m }{\partial Y_k } \nonumber \\
&+ \left( \frac{ a_m}{b_m^2} \frac{1}{\tau+b_m} + \frac{a_m}{b_m} \frac{1}{(\tau+b_m)^2} \right )
\frac{\partial b_m }{\partial Y_j } \frac{\partial b_m }{\partial Y_k }
\end{align}
Note that the first term is ill-behaved as $Y_k \rightarrow 0$ but this is cancelled out by the multiplication by
$\mathcal{Y}$ need to define the $d$.  This cancelation needs to be done analytically for things to work correctly.
One would like to subtract (I think)
\[
\frac{\wbar Y} { (\rho R T)} \nabla p
\]
from $d$ to make them invariant.  
We also need
\begin{align}
\frac{\partial \mu_k}{\partial \tau} &=  
-\frac{RT}{\wbar (\tau-b_m)^2} \frac{\partial b_m}{\partial Y_k}
-\frac{RT}{W_k (\tau-b_m)} \nonumber \\
&+ \left( \frac{\partial a_m}{\partial Y_k} - \frac{a_m}{b_m} \frac{\partial b_m}{\partial Y_k} \right )
\frac{1}{\tau ( \tau+b_m)}
+ \frac{a_m}{b_m} \frac{1}{(\tau - b_m)^2} \frac{\partial b_m}{\partial Y_k}
\end{align}

\section{Species transport -- OLD}
\MarginPar{need to fix this}

For the species transport, we write the chemical potential as
\begin{eqnarray}
\label{eq:muk_sm}
\mu_k &=& \mu_k^{u,id} + \frac{R}{W_k} ln(X_k)  +  \mu_k^{sm} \\
 &=& h_k - T s_k^* + \frac{R T}{W_k} ln(p/p_0) + \frac{R}{W_k} ln(X_k) + \mu_k^{sm}
\end{eqnarray}
This is likely equivalent to using the $\gamma$ if we define them in the right way but I'm not sure
introducing the $ln$ does us much good.
At least according to Giovangigli, $\mu_k^{sm}$ is smooth as species vanish.  For SRk this appears to be true.

In vector form we have (thinking of $\mu$ as a function of $T$, $p$ and $X$
\begin{equation}
\nabla_T \mu
= R T \mathcal{W}^{-1} \mathcal{X}^{-1} \nabla X +
+ R T \mathcal{W}^{-1} \nabla_{T,p} \; \mu^{sm} +
+ \frac{\partial \mu}{\partial p}
\end{equation}
As before we note the
\[
\left(\frac{\partial \mu}{\partial p}\right)_{X,T} \equiv \theta = \left(\frac{ \partial \tau}{\partial Y}\right)_{T,p}
 =
\frac{-1}{\rho^2}\left (\frac{\partial \rho}{\partial Y} \right)_{T,p}
\]
and we define
\[
\label{eq:transp_gradp_multiplier}
\phi = \rho \mathcal{Y} \theta
\]
We then define the thermodynamic driving force
\[
d = \nabla X  + \frac{\overline{W}}{R T} \mathcal{W} \mathcal{Y} \nabla_{T,p} \; \mu^{sm}  + \frac{\mbar}{\rho k_B T} (\phi - Y) \nabla p
\]
and
\[
\mathcal{F} = - \rho \mathcal{Y} \mathcal{D} ( d + \mathcal{X} \frac{ \tilde{\chi}}{T} \nabla T)
\]
and
\[
\HeatFlux = - \lambda \nabla T + k_B T {\tilde{\chi}}^T \mathcal{M}^{-1} \SpeciesFlux + h^T \SpeciesFlux
\]

\subsection{Transport coefficients}

NOTE:  These need more work.

For ideal gas, $\mathcal{D}$ is generated from binary diffusion coefficients, $D_{ij}$ using
Maxwell Stefan relationships.
For dense gases, from kinetic theory of dense gases we modify the binary diffusion coefficent to
\[
\label{eq:Dij}
D_{ij} = \frac{n^{st}}{n \Upsilon_{ij}} D_{ij}^{st}
\]
where $n$ is total particle density and $\Upsilon_{ij}$ is a function of collision diameters.
The $\Upsilon$ are given in \cite{KurochkinETAL:1984}.
See Eq. (35) in \cite{giovangigli_CTM:2011}.

Thermal conductivity can be given as
\MarginPar{Need to figure out how to evaluate the things appearing here - and to determine if they matter - Shashank making plots from NIST to see if they do}
\[
\lambda = \lambda^{dil} \beta + \lambda^{hp}
\]
from Chung et al. \cite{chung:1988} and Ely and Hanley \cite{ElyHanley:1983}

Giovangigli uses dilute thermal diffusion rations in 
\cite{giovangigli_CTM:2011}.

From BCH and Chapman and Cowling,
for single component gases define
\[
\beta = \frac {2}{3} \pi n \sigma^3 \;\;\; \mathrm{and} \;\;\; \zeta = 1+\frac{5}{8} \beta
\]
where $\sigma$ is molecular diameter.
Then for dense gases they define
\[
\eta = \eta_0 (\frac{1}{\zeta} + 0.8 \beta + 0.761 \beta^2 \zeta)
\]
where $\eta_0$ is the low pressure shear viscosity.

They also set bulk viscosity as
\[
\kappa = 1.002 \eta_0 \beta^2 \zeta
\]
This has rather a hard sphere quality

For a monatomic gas they also set
\[
\lambda = \lambda_0 (\frac{1}{\zeta} + 1.2 \beta + 0.755 \beta^2 \zeta)
\]
but this is not supposed to be good for polyatomic molecules.
For polyatomic they suggest
\[
\lambda = \eta_0 \frac{R}{W \zeta} \left(\frac{c_v}{k_b} + \frac{9}{4} \right)
+ \frac{15 R}{4 w} \eta_0 \beta \left( 1.2 + 0.755 \beta \zeta \right ) 
\]

Need to look for mixture formulae.  Suggest we consider some of the formulae in Poling book that Shashank
recommended.

\bibliographystyle{unsrt}% Produces the bibliography via BibTeX.
\bibliography{nonideal_refs}% Produces the bibliography via BibTeX.

\end{document}
