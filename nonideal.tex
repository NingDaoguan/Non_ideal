\documentclass[11pt]{article}

\usepackage{graphicx}% Include figure files
\usepackage{dcolumn}% Align table columns on decimal point
\usepackage{bm}% bold math
%\usepackage{hyperref}% add hypertext capabilities
%\usepackage[mathlines]{lineno}% Enable numbering of text and display math
%\linenumbers\relax % Commence numbering lines

%\usepackage[showframe,%Uncomment any one of the following lines to test
%%scale=0.7, marginratio={1:1, 2:3}, ignoreall,% default settings
%%text={7in,10in},centering,
%%margin=1.5in,
%%total={6.5in,8.75in}, top=1.2in, left=0.9in, includefoot,
%%height=10in,a5paper,hmargin={3cm,0.8in},
%]{geometry}

\usepackage{subfigure}
\usepackage{longtable}
\usepackage{lipsum}
\usepackage{color}
\usepackage{amsmath}
\usepackage{amssymb}


\newcommand{\HeatFlux}{\boldsymbol{\mathcal{Q}}}
\newcommand{\SpeciesFlux}{\boldsymbol{\mathcal{F}}}
\newcommand{\SpeciesFluxB}{\bar{\boldsymbol{\mathcal{F}}}}
\newcommand{\StressTensor}{\boldsymbol{\Pi}}
\newcommand{\ShearViscosity}{\eta}
\newcommand{\BulkViscosity}{\kappa}
\newcommand{\ThermalConductivity}{\lambda}
\newcommand{\EntropyProduction}{\mathfrak{v}}
\newcommand{\OnsagerMatrix}{\boldsymbol{\mathfrak{L}}}
\newcommand{\OnsagerMatrixB}{\bar{\boldsymbol{\mathfrak{L}}}}
\newcommand{\OnesVector}{\mathbf{u}}
\newcommand{\ChemicalSpeciesLabel}{\mathfrak{M}}
\newcommand{\StochioPM}{\nu}

\newcommand{\mbar}{\overline{m}}
\newcommand{\Jbar}{\bar{\mathbf{J}}}
\newcommand{\Xbar}{\bar{\mathbf{X}}}
\newcommand{\Lbar}{\bar{\mathbf{L}}}
\newcommand{\Bbar}{\bar{\mathbf{B}}}
\newcommand{\Wbar}{\bar{\mathcal{Z}}}
\newcommand{\Wcbar}{\bar{\mathcal{W}}}
\newcommand{\Bcbar}{\bar{\mathcal{B}}}
\newcommand{\lbar}{\bar{\mathbf{l}}}
\newcommand{\xibar}{\bar{\xi}}
\newcommand{\DonevDiffusion}{{D}}
\newcommand{\half}{\frac{1}{2}}

\begin{document}
\begin{center}
{\bf
Multicomponent reacting compressible Navier Stokes for nonideal gases
}

\vspace{\baselineskip}
John B. Bell \\
Alejandro Garcia \\
Ray Grout \\
Emmanuel Motheau \\
Andrew Nonaka \\
Shashank

\vspace{\baselineskip}
April 5, 2017
\end{center}

\section{Equations}

As a prelude, much of the later parts of this that are specific to SRK equation of state are taken from 
paper by Giovangigli and collaborators \cite{giovangigli_CTM:2011,giovangigli:2012}.

The continuity, species, momentum and energy equations are:
\begin{equation}
\frac{\partial }{\partial t} \left( \rho \right)  + { \nabla} \cdot \left( \rho { u} \right) = 0,
\label{eqn:cont}
\end{equation}
\begin{equation}
\frac{\partial }{\partial t} \left( \rho Y \right)  + { \nabla} \cdot \left( \rho { u} Y \right) + { \nabla} \cdot
\SpeciesFlux
=  \omega 
\label{eqn:spec}
\end{equation}
\begin{equation}
\frac{\partial }{\partial t} \left( \rho { u} \right)  + { \nabla} \cdot \left( \rho { u \otimes u } \right) + { \nabla} p + {\nabla} \cdot   \StressTensor = 0,
\label{eqn:mom}
\end{equation}
\begin{equation}
\frac{\partial }{\partial t} \left( \rho E \right)  + { \nabla} \cdot \left( \rho { u} E + p { u}  \right) + { \nabla} \cdot  { \HeatFlux}  + \nabla \cdot (\StressTensor u)
\label{eqn:energy}
\end{equation}
where $\rho$, ${u}$ and $p$ denote the density, velocity vector and pressure, respectively, of the mixture.
The mass fraction of the $k$-th species is given by $Y_k$, and is denoted in vector form as $Y$.
The different species in the gaseous mixture are assumed to be in thermal equilibrium,
that is, at a common temperature, $T$.

In the momentum equation, the tensor $\StressTensor$ is the viscous stress and, under the Newtonian assumption, is given by:
\begin{equation}
\StressTensor = \eta ( \nabla u + ( \nabla u )^T) + ( \kappa - \frac{2}{3} \eta) \mathbf{I} \nabla \cdot u
\end{equation}

The formulation of the multi-species stochastic diffusion and heat fluxes is complicated by the couplings among the species fluxes (cross-diffusion effects) and by the thermal diffusion contribution (Soret and Dufour effects).
The starting point for determining these fluxes
is the entropy production for a mixture, as formulated by de Groot and Mazur~\cite{DM_63} and by Kuiken~\cite{NewIrrevThermoBook},
which establishes the form of the thermodynamic forces and fluxes.
The entropy production also has a contribution due to the stress tensor, however,
due to the Curie symmetry principle \cite{DM_63}, fluxes and thermodynamic forces of different tensorial
character do not couple.

%Following Eqn. (IV.13) in De Groot and Mazur,
The entropy production for a
multi-component mixture at rest, in the absence of external forces~\footnote{This contribution
is also zero if the external specific force
acting on each species is constant, as with a constant gravitational acceleration.} and chemistry,
is given by \cite{DM_63}:
%
\begin{eqnarray}
\EntropyProduction &=&
-\frac{1}{T^2} \HeatFlux' \cdot \nabla T
- \frac{1}{T} \sum_{i=1}^{N_s} \SpeciesFlux_i \cdot \nabla_T \mu_i \\
&=& -\frac{1}{T^2} \HeatFlux' \cdot \nabla T
- \frac{1}{T} \sum_{i=1}^{N_s-1} \SpeciesFlux_i \cdot \nabla_T \left(\mu_i - \mu_{N_s} \right ),
\label{eq:dGM1}
\end{eqnarray}
%
where $\mu_i$ is the chemical potential per unit mass of species $i$ and
\begin{equation}
\HeatFlux' = \HeatFlux - \sum_{k=1}^{N_s} h_k \SpeciesFlux_k
= \HeatFlux - \sum_{k=1}^{N_s-1} (h_k-h_{N_s}) \SpeciesFlux_k,
\end{equation}
where $h_k $ is the specific enthalpy of the $k^{th}$ component.
In other words, $\HeatFlux'$ is the part of the heat flux that is \emph{not} associated with mass diffusion.
Here, $\nabla_T$ is a gradient derivative taken holding temperature fixed, that is,
\[
\nabla_T  \; \mu_i(p,T,X_1, \ldots, X_{N_s-1}) = \nabla \mu_i -
\left(\frac{\partial \mu_i}{\partial T}\right)_{p,X_1, \ldots, X_{N_s-1}}\, \nabla T,
\]
where $X_k=n_k/\sum_{j=1}^{N_s} n_j$ are mole fractions, and $n_k$ are number densities.
The mole fraction for species $k$ is given in terms of the mass fractions by $X_k = ({\overline{m}}/{m_k}) Y_k$,
where $m_k$ is the
mass of a molecule of that species, and $\overline{m}=\left( \sum_{k=1}^{N_s}Y_k / m_k \right)^{-1}$ is the mixture-averaged molecular weight \cite{NewIrrevThermoBook}.  Note that only $N_s-1$ of the mass or mole fractions are independent.
%
%This form of the heat flux arises from using $\nabla_T \, \mu_i$ instead of the full gradient (and a thermodynamics relation).~\textbf{Add citation or clarify.}
% note that the sum here is over \emph{all} species.


The general form of the phenomenological laws
expresses the fluxes as linear combinations of thermodynamics forces, written in matrix form as
\[
\Jbar = \OnsagerMatrixB \Xbar
\qquad\mathrm{where}\qquad
\EntropyProduction = \Jbar^T \Xbar =  \Xbar^T \OnsagerMatrixB^T \Xbar.
\]
Here we use an overbar to denote the system expressed in terms of the first $N_s-1$ species.
From (\ref{eq:dGM1}) the fluxes $\Jbar$ and the thermodynamics forces $\Xbar$ are given by
\[
\Jbar =
\begin{bmatrix} \SpeciesFluxB \\ \HeatFlux'
\end{bmatrix}
\qquad \mathrm{and} \qquad
\Xbar = \begin{bmatrix}
- \frac{1}{T} \nabla_T ( \mu_i - \mu_{N_s})
\\
- \frac{1}{T^{2}}\nabla T
\end{bmatrix}
\]
respectively,  where $\SpeciesFluxB = [\SpeciesFlux_1,\ldots,\SpeciesFlux_{N_s-1}]^T$ is a vector of $N_s-1$ independent species mass fluxes.
By Onsager reciprocity the matrix of phenomenological coefficients is symmetric so we can write $\OnsagerMatrixB$ as
\[
\OnsagerMatrixB =
\begin{bmatrix}
{\Lbar} & \lbar
\\
{\lbar}^T & \ell
\end{bmatrix}  \;\;\;  ,
\]
where $\Lbar$ is a symmetric $N_s-1 \times N_s-1$ matrix that depends on the multicomponent flux diffusion coefficients, $\lbar$ is an $N_s-1$ component
vector that depends on the thermal diffusion coefficients, and the scalar $\ell$ depends on the
partial thermal conductivity.

It is useful to recast $\OnsagerMatrixB$ in a slightly different form.
This form will facilitate comparison with the continuum transport literature (e.g.,~\cite{Giovangigli_99})  and lead
to a more efficient numerical algorithm.
We introduce
\[
\xibar = \Lbar^{-1} \lbar
\qquad \mathrm{and} \qquad
\zeta = \ell - \xibar^T \Lbar \xibar
\]
so that
\begin{equation}
{\OnsagerMatrixB} =
\begin{bmatrix}
{\Lbar} & {\Lbar \xibar}
\\
{\xibar^T{\Lbar}} & \zeta + \xibar^T {\Lbar} \xibar
\end{bmatrix}.
\label{eq:Lbar_xi}
\end{equation}
It is important to point out that this construction works even when $\Lbar$ is not invertible,
which happens when some of the species are not present. This is because $\xibar$ is always in the range of $\Lbar$.

\subsection{Full System Construction}

The form of the equations above requires that we distinguish a particular species, numbered $N_s$, which must be present
throughout the entire system. For many applications, this introduces an artificial requirement on the system
that is difficult to deal with numerically.
In this section we transform the reduced form with $N_s-1$ equations, used by de Groot and Mazur, to an equivalent full system construction.
It is noted in de Groot and Mazur that the Onsager reciprocal relations remain valid in the presence
of linear constraints such as ($\sum Y_k = 1$).
In particular, we can consider the full system with $N_s+1$ equations (including thermal diffusion) with
the constraint $\sum_k \mathcal{F}_k  = 0$
by defining an augmented system that gives exactly the same entropy production.
In particular, we define an augmented Onsager matrix $\mathbf{L}$ of the form
\[
{\mathbf{L}} =
\begin{bmatrix}
\Lbar & -{\Lbar}\OnesVector \\
- \OnesVector^T {\Lbar} & \OnesVector^T\Lbar \OnesVector
\end{bmatrix}
\]
where $\OnesVector = [1,\ldots,1]^T$.  Here the final row gives $\SpeciesFlux_{N_s}$, the diffusion flux of the last species.
The extra row and column of ${\mathbf{L}}$ are fully specified by the
requirement that column sums vanish (a consequence of vanishing of the sum of
species fluxes) and the Onsager symmetry principle.

Using ${\mathbf{L}}$ we can write the phenomenological laws
for the full system as
\[
{\mathbf{J}} = {\OnsagerMatrix} {\mathbf{X}},
\]
where the fluxes ${\mathbf{J}}$ and thermodynamics forces ${\mathbf{X}}$ are given by
\[
{\mathbf{J}} =
\begin{bmatrix}
{\SpeciesFlux}
\\
\HeatFlux'
\end{bmatrix}
\qquad \mathrm{and} \qquad
{\mathbf{X}} = \begin{bmatrix}
- \frac{1}{T} \nabla_T  \mu
\\
-\frac{ \nabla T }{ T^2 }
\end{bmatrix}
\]
with
\begin{equation}
{\OnsagerMatrix} =
\begin{bmatrix}
{{\mathbf{L}}} & {{\mathbf{L}} {\xi}}
\\
{\xi}^T{{\mathbf{L}}} & \zeta + {\xi}^T {{\mathbf{L}}} {\xi}
\end{bmatrix}.
\label{eq:L_xi}
\end{equation}
Ottinger \cite{Ottinger_09} gives a derivation of this
form using the GENERIC formalism subject to the linear constraint $\sum_{k=1}^{N_s} Y_k=1$.

A direct computation shows that  gives
\[
\EntropyProduction
= \Jbar^T \Xbar
= {\mathbf{J}}^T {\mathbf{X}}
=  {\mathbf{X}}^T {\OnsagerMatrix} {\mathbf{X}}.
\]
Hence the full system form gives exactly the same entropy production as the original form.

From (\ref{eq:L_xi}) we can then obtain the species flux
\begin{equation}
\SpeciesFlux = -\frac{1}{T} \mathbf{L} \left [ \nabla_T \mu + \frac{\xi}{T}\nabla T \right ]
\label{eq:F_Onsager}
\end{equation}
and the deterministic heat flux
\begin{equation}
\HeatFlux
= -\zeta \frac{\nabla T}{T^2}  +
(\xi^T + \mathbf{h}^T) \SpeciesFlux,
\label{eq:Q_Onsager}
\end{equation}
where $\mathbf{h}$ is the vector of specific enthalpies.

\section{Nonideal systems}

For nonideal fluids we have
\[
\mu_k(X,T,p) = \mu_k^0(T,p) + \frac{k_B T}{m_k} ln (X_k) + \frac{k_B T}{m_k} ln (\gamma_k)
\]
Note:  John notation:  for a quantify such as $Y$, $Y$ is the vector of the quantity, $Y_k$ is an
element and $\mathcal{Y}$ is a diagonal matrix of $Y$. For example,
\[
\frac{m_k X_k}{\mbar} = Y_k \leftrightarrow \mathcal{M} X = \mbar Y
\]
In vector form then
\[
\mu_(X,T,p) = \mu^0(T,p) + {k_B T}{\mathcal{M}^{-1}} ln (X_k) + {k_B T}\mathcal{M}^{-1} ln (\gamma)
\]
Then
\begin{eqnarray}
\nabla_T \mu
&=& k_b T \mathcal{M}^{-1} \mathcal{X}^{-1} \nabla X +
 k_b T \mathcal{M}^{-1} Diag(1/\gamma_k) \frac{\partial \gamma}{\partial X} \nabla X + \frac{\partial \mu}{\partial p} \\
&=& k_b T \mathcal{M}^{-1} \mathcal{X}^{-1} \Gamma \nabla X + \frac{\partial \mu}{\partial p}
\end{eqnarray}
where $\Gamma = I + \frac{\partial ln (\gamma)}{\partial ln(X)}$.
Also
\[
\frac{\partial \mu}{\partial p} \equiv \theta = \frac{ \partial \tau}{\partial Y} =
\frac{-1}{\rho^2}\frac{\partial \rho}{\partial Y}
\]
where $\tau$ is specific volume. The second equality is a Maxwell relation.

We don't want to work with this form because $\nabla_T \mu$ is badly behaved when species vanish.
Introduce the thermodynamic driving force
\[
d = \Gamma \nabla X  + \frac{\mbar}{\rho k_B T} (\phi - Y) \nabla p
\]
where $\phi = \rho \mathcal{Y} \theta$.  The $Y$ term in the $\nabla p$ part makes thing frame invariant.
Note that
\[
k_B T \mathcal{M}^{-1} \mathcal{X}^{-1} \frac{\mbar}{\rho k_B T} \phi \nabla p = 
k_B T \mathcal{M}^{-1} \mathcal{X}^{-1} \frac{\mbar}{\rho k_B T} \rho \mathcal{Y} \theta \nabla p 
= \theta \nabla p
\]
Following Giovangigli,
the multicomponet diffusive fluxes are then given by
\begin{eqnarray}
\SpeciesFlux &=& -\rho \mathcal{Y} \mathcal{D} ( d + \frac{\chi}{T} \nabla T) \\
&=& -\rho \mathcal{Y} \mathcal{D} ( \Gamma \nabla X  + \frac{\mbar}{\rho k_B T} (\phi - Y) \nabla p
 + \frac{\chi}{T} \nabla T)
\end{eqnarray}
It is conventient to write $\chi = \mathcal{X}\tilde{\chi}$ so
\[
\SpeciesFlux = -\rho \mathcal{Y} \mathcal{D} ( \Gamma \nabla X  + \frac{\mbar}{\rho k_B T} (\phi - Y) \nabla p
 + \mathcal{X}\frac{\tilde{\chi}}{T} \nabla T)
\]
and
\[
\HeatFlux = - \lambda \nabla T + k_B T {\tilde{\chi}}^T \mathcal{M}^{-1} \SpeciesFlux + h^T \SpeciesFlux
\]
To link this back to Onsager form
\[
L = \frac{\rho \mbar}{k_B} \mathcal{Y} \mathcal{D} \mathcal{Y}
\]
\[
\xi = k_B T \mathcal{M}^{-1} \tilde{\chi}
\]
\[
\zeta = T^2 \lambda
\]

This general formalism needs to be linked the to the specific form of the EOS we later use.

\section{Summary of what we need to integrate compressible flow equations}

In order to solve the system, we need to be able to evaluate a number of different things.
This is an attempt to start a list
\begin{itemize}
\item Evaluate $p$ given $\rho$, $T$ and $Y_k$. This implicitly involves whatever mixing rules are used to
evaluate repulsion and attraction terms in cubic EOS
\item Evaluate sound speed, basically $\partial ln(p)/ \partial ln(\rho)$ at constant entropy and $Y_k$. This
can be expressed in terms of other derivatives.
\item Need matrix of diffusion coefficients $\mathcal{D}$.  This typically requires binary diffusion coefficients and may involve high density corrections
\item The matrix $\Gamma$ that represents nonideal chemical potential effect on transport
\item The vector of $\theta$'s (or $\phi$'s) that represent volume fractions in barodiffusion
\item Need matrix of rescaled thermal diffusion ratios $\tilde{\chi}$ (if you want Soret and Dufour)
\item Thermal conductivity $\lambda$ and viscosity coefficients $\eta$ and $\kappa$
\item The species enthalpies $h_k$ that are need to form energy diffusive flux
\item The species chemical potentials that are needed for chemistry
\item A specification of chmical rates
\item One probably needs specific heats, at least to compute thermal diffusion time step limit
\end{itemize}

\section{One approach to make this specific}

What we would like to do is start with an equation of state and derive other quantities we need that
are thermodynamically consistent.  The EOS doesn't nail things down completely.  A way to get around that
(which is done by Giovangigli in his nonideal papers) is to
say
\[
p = p^{id} + \Phi
\]
where $p^{id}$ is the ideal gas part and assume that as density goes to zero, the EOS becomes
ideal.  In other words we look at deviation form ideal.
What Giovangigli does is to write thermodynamic quantities as
\[
\Psi = \Psi^{id} - \int_\tau^\infty \frac{\partial \Psi}{\partial \tau} d \tau
\]

For internal energy from the Helmholtz equation
\begin{equation}
\frac{\partial e_m}{\partial \tau} = T^2 \frac{\partial (\Phi/T)} {\partial T}
\end{equation}
If we use the above and note that $p^{id}/T$ is independent of $T$ then
\begin{equation}
\frac{\partial e_m}{\partial \tau} = e^{id}- \int_\tau^\infty  T^2 \frac{\partial (\Phi/T)} {\partial T} d\tau
\label{eq:energy_gen}
\end{equation}

For entropy of mixture, $s_m$,
start with free energy $f = e_m - Ts_m$.
We know that
\[
df = -s dT - p d\tau + \sum \mu_k dY_k
\]
and for ideal EOS
\[
df^{id} = -s^{id} dT - p^{id} d\tau + \sum \mu_k^{id} dY_k
\]
Then
\begin{equation}
\frac{\partial f}{\partial \tau} = -p \;\; \mathrm{and} \;\;\;
\frac{\partial f^{id}}{\partial \tau} = -p^{id}
\end{equation}
so
\[
\frac{\partial (f^{id} - f)}{\partial \tau} = \Phi
\]
Then
\begin{equation}
(f^{id} - f)(\tau) - 
(f^{id} - f)(\tau '' ) = -\int_\tau^{\tau ''} \Phi (\tau ')d\tau ' 
\end{equation}
As $\tau'' \rightarrow \infty$, $( f^{id}-f) \rightarrow 0$
so
\begin{equation}
(f^{id} - f)(\tau) 
 = -\int_\tau^{\infty} \Phi (\tau ')d\tau ' 
\end{equation}
or
\begin{equation}
(e_m^{id} - e_m)(\tau) - 
T(s_m^{id} - s_m)(\tau) 
 = -\int_\tau^{\infty} \Phi (\tau ')d\tau ' 
\end{equation}
Rearranging
\begin{equation}
s_m = s_m^{id} + \frac{e_m-e_m^{id}}{T} - \frac{1}{T}
\int_\tau^{\infty} \Phi (\tau ')d\tau ' 
\end{equation}
If we now use (\ref{eq:energy_gen})
we get
\begin{eqnarray}
s_m &=& s_m^{id} -
\int_\tau^\infty  T^2 \frac{\partial (\Phi/T)} {\partial T} d\tau
- \frac{1}{T}
\int_\tau^{\infty} \Phi (\tau ')d\tau '  \\
&=& s^{id}
-\int_\tau^{\infty} \frac{\partial (\Phi)} {\partial T} d\tau ' 
\end{eqnarray}

From free energy we also have
\[
\mu_k = \frac{\partial f}{\partial Y_k} = 
 \frac{\partial e_m}{\partial Y_k} - 
 T\frac{\partial s_m}{\partial Y_k} 
\]
and
\[
\mu_k^{id} = \frac{\partial f^{id}}{\partial Y_k} = 
 \frac{\partial e_m^{id}}{\partial Y_k} - 
 T\frac{\partial s_m^{id}}{\partial Y_k} 
\]
As an aside, for ideal gases, $\mu_k = h_k + T s_k$.  It is worth noting for ideal gases 
that
\[
\mu_k = \partial_{Y_k} (e - T s) = 
 \partial_{Y_k} \sum Y_k(e_k - T s_k) = e_k - T s_k - T Y_k \partial_{Y_K}s_k = h_k + T s_k
\]

Using the formulas above
\begin{eqnarray}
\mu_k &=& \mu_k^{id} - \frac{\partial}{\partial Y_k} \int_\tau^\infty
 T^2 \frac{\partial (\Phi/T)} {\partial T} - T \frac{\partial (\Phi)} {\partial T} d\tau \\
 &=& \mu_k^{id} + \int_\tau^\infty \frac{\partial (\Phi)} {\partial Y_k} d\tau
\end{eqnarray}
Note that this is the opposite sign to Giovangigli.

There is, what appears to be, a more straight forward way to derive these results
but they seem to retain $p$ terms in the integrals
and the integrals are not convergent. Not sure how to reconcile those with these arguments.

%
%Similarly
%\[
%\frac{\partial s_m}{\partial \tau} = \frac{\partial (e-f)/T}{\partial \tau} = \frac{\partial \Phi}{\partial T}
%\]
%where $f$ is the Helmholtz free energy with $\frac{\partial f}{\partial \tau} = -\Phi$
%
%A similar thing is supposed to work for species chemical potential but it relies on
%\[
%\frac{\partial \mu_k}{\partial \tau} = \frac{\partial \Phi}{\partial Y_k} 
%\]
%and I don't know where that comes form yet.
%
An issue is that it gives things as function of $\rho$, $T$ and $Y$.  One of the
things we need to construct the energy flux is $h_k$, the species enthalpies.
Those are given by 
\[
h_k = \left. \frac{\partial h}{\partial Y_k} \right|_{T,p}
\]
The natural form given by the Giovangigli approach gives things as a function of $\rho$, $T$ and $Y_k$.
This can be address using
\[
\frac{\partial h}{\partial (T,Y,p)}
= \frac{\partial h}{\partial (T,Y,\rho)}
\frac{\partial (T,Y,\rho)}{\partial (T,Y,p)}
= \frac{\partial h}{\partial (T,Y,\rho)}
\left (\frac{\partial (T,Y,p)}{\partial (T,Y,\rho)} \right)^{-1}
\]

It turns out as an aside the the low Mach number constraint for general EOS also relies on
write $h_m$ as a function of $p$, $T$, and $Y$.

\section{Kinetics}

For kinetics you work with rescaled chemical potentials
\[
\hat{\mu} = \frac{1}{k_B T} \mathcal{M} \mu
\]
Molar production rates are given by
\[
\Omega_k = \sum_j (\nu_{ij}^b - \nu_{ij}^f) r_j
\]
where 
\[
r_j = k_j^s \left [ exp(\nu_{ij}^f \hat{\mu}_i) - exp ( \nu_{ij}^f \hat{\mu}_i) \right ] 
\]
where $k_j^s$ is the symmetric reaction rate.

To express this in the more traditional form, we can define backward rates in terms of forward rates
and equilibrium coefficients.  Giovangigli asserts that ideal equilibrium constants are ok as 
reference points but I'm not sure about that.

If one can express the chemical potentials in terms of the activity coefficients then the 
kinetics can be expressed in terms of a generalized law of mass action that involves the $\gamma$'s.

\section{Transport coefficients}

For ideal gas, $\mathcal{D}$ is generated from binary diffusion coefficients, $D_{ij}$ using
Maxwell Stefan relationships.
For dense gases, Giovangigli advocates
\[
D_{ij} - \frac{n^{st}}{n \Upsilon_{ij}} D_{ij}^{st}
\]
where $n$ is total particle density and $\Upsilon_{ij}$ is a function of collision diameters.

\section{Soave-Redlich-Kwong}

Soave-Redlich-Kwong is one of the cubic equations of state that represent generalizations of
van der Waal's EOS.
(Peng-Robinson is similar).

SRK is given by
\[
p = k_B T \sum \frac{Y_k}{m_k} \frac{1}{\tau - \bar{b}} - \frac{\bar{a}}{\tau(\tau + \bar{b})}
\]
\[
p^{id} = k_B T \sum \frac{Y_k}{m_k} \frac{1}{\tau}
\]
\[
\Phi_{SRK}
= k_B T \sum \frac{Y_k}{m_k} \frac{\bar{b}}{\tau(\tau -\bar{b})} - \frac{\bar{a}}{\tau(\tau + \bar{b})}
\]
Here $\bar{a} = \bar{a}(T, Y_k)$ and $\bar{b} = \bar{b}(Y_k)$
\[
\bar{a} = (\mathcal{Y}\sqrt(a))^T  \mathcal{Y}\sqrt(a) \;\;\;  \bar{b} = Y^T b
\]
were $a$ is the vector of species attractive parameters that are a function of temperature and are
set in relation to the critical temperature and $b$ is the vector species repulsive forces which are
constants (I think) that depend on critical point.
See Giovangigli papers for form of these.

Giovangigli carries out the derivation of some of the other quantities for SRK. For example,
\[
e = Y^T e^{id} + ( T \left . \frac{\partial \bar{a}}{\partial T} \right |_{\rho,Y} - \bar{a})
\frac{1}{\bar{b}} ln ( 1 + \frac{\bar{b}}{\tau})
\]
and
\[
h = Y^T h^{id} + ( T \left . \frac{\partial \bar{a}}{\partial T} \right |_{\rho,Y} - \bar{a})
\frac{1}{\bar{b}} ln ( 1 + \frac{\bar{b}}{\tau})
+ 
k_B T \sum \frac{Y_k}{m_k} \frac{\bar{b}}{\tau -\bar{b}} - \frac{\bar{a}}{\tau(\tau + \bar{b})}
\]

Tragically, this doesn't work for Peng-Robinson.  The integral doesn't appear to have a nice closed from solution, at leat as far as I can
see.

\section{Departure Functions}
Following derivations presented in the book by Poling~\cite{poling2001properties} we can derive generalized expressions for functions representing departure from ideal gas behaviors. In all the equations listed below the departure functions are presented in the form $F^{d} = F^{ig} - F^{rg}$ where the superscripts $ig$ and $rg$ denote ideal gas and real gas respectively. 
\begin{enumerate}
\item Compressibility factor: 1 - Z
\item Internal energy e:  
\begin{equation}
\label{eq:DepartInternalEnergyIntegral}
\frac{e^{ig}-e}{RT} = \int_{V}^{\infty} \left[ T \left(\frac{\partial Z}{\partial T}\right)_{V} \right] \frac{dV}{V}
\end{equation}
\item Enthalpy h: 
\begin{equation}
\frac{h^{ig}-h}{RT} = \frac{e^{ig}-e}{RT}  + 1 - Z
\end{equation}
\item Helmholtz energy A: 
\begin{equation}
\frac{A^{ig}-A}{RT} = \int_{V}^{\infty} \left[1-Z\right]\frac{dV}{V} + ln Z
\end{equation}
\item Entropy s:
\begin{equation}
\frac{s^{ig}-s}{RT} = \frac{e^{ig}-e}{RT}  -  \frac{A^{ig}-A}{RT}
\end{equation}
\item Gibbs energy, G:
\begin{equation}
\frac{G^{ig}-G}{RT} = \frac{A^{ig}-A}{RT} + 1 - Z
\end{equation}
\end{enumerate}
A generalized cubic EOS in terms of compressibility factor, Z can be written as 
\begin{equation}
Z = \frac{V}{V-B_{m}} - \frac{\left(\Theta/RT\right) V}{\left(V^{2} - \delta V + \epsilon \right)}
\end{equation}
where for
\begin{enumerate} 
\item Redlich-Kwong (RK) EOS: $\delta = B_{m}$, $\epsilon = 0$, $\Theta = A_{m}/\sqrt{T_{r}}$ 
\item Peng-Robinson (PR) EOS: $\delta = 2 B_{m}$, $\epsilon = -B_{m}^2$, $\Theta = A_{m} \alpha(T_{r})$. 
\end{enumerate} 
Using the above presented form of EOS and departure functions defined in Eq.~\ref{eq:DepartInternalEnergyIntegral}, exact form of departure functions for cubic EOS can represented as the following 
\begin{enumerate}
\item Internal energy e, RK: 

\begin{equation}
\left(\frac{\partial Z}{\partial T}\right)_{V} = \frac{3}{2} \frac{A_{m}}{R T^{2} \left[V+B_{m} \right]}
\end{equation}

\begin{align}
e^{ig} - e &= \frac{3 A_{m}}{2} \int_{V}^{\infty} \frac{dV}{V^{2}+V B_{m}} \\
               & = \frac{3 A_{m}}{2 B_{m}} ~ln \left[\frac{V}{V+B_{m}}\right]
\end{align}
In order to simplify relationships for other quantities, a quantity $K_{1}$ will be defined
\begin{equation}
K_{1} = \frac{1}{B_{m}} ~ln \left[\frac{V}{V+B_{m}}\right]
\end{equation}
\item Internal energy e, PR:

\begin{equation}
\left(\frac{\partial Z}{\partial T}\right)_{V} = \left[ \frac{-V}{V^{2}+ 2V B_{m} - B_{m}^{2}}\right]\left[ \frac{\frac{\partial A_{m}}{\partial T}T -A_{m}}{R T^{2}}\right]
\end{equation}
\begin{align}
e^{ig} - e & = T \int_{V}^{\infty} \left[ \frac{-1}{V^{2}+2 V B_{m}-B_{m}^{2}}\right] \left[ \frac{\partial A_{m}}{\partial T} - \frac{A_{m}}{T}\right] dV \\
               & = \left[ -T \frac{\partial A_{m}}{\partial T} + A_{m}\right] \int_{V}^{\infty} \frac{dV}{\left[V+(1-\sqrt{2}) B_{m} \right] \left[ V+(1+\sqrt{2}) B_{m} \right]} \\
               & = \left[ T \frac{\partial A_{m}}{\partial T} - A_{m}\right] \frac{1}{\sqrt{2}B_{m}} tanh^{-1}\left(\frac{V+B_{m}}{\sqrt{2}B_{m}}\right)
\end{align}
Similar to the case with RK EOS, a quantity $K_{1}$ can be defined 
\begin{equation}
K_{1} = \frac{1}{\sqrt{2}B_{m}} tanh^{-1}\left(\frac{V+B_{m}}{\sqrt{2}B_{m}}\right)
\end{equation}
For both RK EOS and PR EOS the internal energy departure function can be written as 
\begin{equation}
e^{ig} - e = \left[ T \frac{\partial A_{m}}{\partial T} - A_{m}\right] K_{1}
\label{eq:internalEnergyFinal}
\end{equation}
\item Enthalpy h: 
For both RK EOS and PR EOS
\begin{equation}
h^{ig} - h = \left[ T \frac{\partial A_{m}}{\partial T} - A_{m}\right] K_{1} + (1 - Z) RT
\end{equation}
\item Specific heat at constant Pressure, Cp: In order to calculate Temperature from enthalpy we would need an expression for Cp. In order to derive this expression we would rely on the equation for internal energy, Eq.~\ref{eq:internalEnergyFinal}
\begin{equation}
C_{V} = \left(\frac{\partial e}{\partial T}\right)_{V} = C_{v}^{ig} - K1 \left[T \frac{\partial^{2} A_{m}}{\partial T^{2}} \right]
\end{equation}
Using the following relation
\begin{equation}
C_{P} = C_{V}  
\end{equation}
In the case of ideal gas the above expression is equivalent to 
\begin{equation}
C_{P}^{ig} = C_{V}^{ig} - R
\end{equation}
Thus the final expression for $C_{P}$ is 
\begin{equation}
C_{P} = C_{P}^{ig} -R -K1 \left[T \frac{\partial^{2} A_{m}}{\partial T^{2}} \right] - T\left(\frac{\partial P}{\partial T}\right)^{2}_{V}/ \left(\frac{\partial P}{\partial V}\right)_{T}
\end{equation}
\end{enumerate}

\bibliographystyle{unsrt}% Produces the bibliography via BibTeX.
\bibliography{nonideal_refs}% Produces the bibliography via BibTeX.

\end{document}
